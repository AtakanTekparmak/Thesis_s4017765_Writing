\section{Introduction}\label{sec:introduction}

Adaptive agents, for example humans, tend to explore new policies to try to outperform opponents in competitive settings. One possible approach is to reason about the unrevealed mental content of the other agents and act accordingly to their state of mind. This method of reasoning about the others' mental state is called \textit{Theory of Mind} \citep{premack1978does}, which has been found to be effective in terms of social competence \citep{liddle2006higher} and other settings like negotiation \citep{de2017negotiating} and the \textit{Mod Game} \citep{veltman2019training}, which will be further discussed in this paper.

Theory of Mind (ToM) is the ability to understand that other people have their own thoughts, beliefs, and intentions that may be different from one's own. It is a key component of social cognition, which is the ability to understand and interpret the actions and intentions of others. Theory of Mind is a crucial skill for successful social interactions, as it allows individuals to take the perspective of others, predict their behavior, and respond appropriately. Theory of Mind is a skill that develops over time, and is typically fully developed by the age of 5 \citep{call2008does}. However, it may not be limited to humans, and has been observed in other species such as chimpanzees \citep{premack1978does}.

In competitive settings, Theory of Mind can be an important factor in predicting and understanding the behavior of one's opponents. This is also supported by the Machiavellian Intelligence Hypothesis, which states that the evolution of intelligence is driven by the need to outwit others, as social success (achieved mainly using methods such as deception and manipulation, which requires a higher level of Theory of Mind) leads to reproductive success \citep{gavrilets2006dynamics}, which is one of the most driving factors of evolution. For example, individuals with a well-developed Theory of Mind may be able to anticipate their opponents' moves in a game of Rock-Paper-Scissors, giving them an advantage over their opponents who use Theory of Mind to a lesser extent\citep{de2013much}. Additionally, individuals with a strong Theory of Mind may be better able to interpret their opponent's non-verbal cues and respond accordingly. 

Overall, Theory of Mind is an important aspect of social cognition that can play a significant role in competitive settings not only for humans but for agents as well. Theory of Mind can also be applied recursively, in the sense that an agent can reason about the Theory of Mind of another agent, and so on. This is sometimes mentioned as \textit{Higher Order Theory of Mind}, where each consecutive order adds another layer of reasoning on top of the previous one. In \cite{de2013much} it is shown that in many different competitive settings, higher orders of Theory of Mind have a competitive advantage over lower levels, but it should be noted that the competitive advantage between levels decreases as the levels go up and tests were done for only up to the fourth level Theory of Mind.
