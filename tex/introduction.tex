\section{Introduction}\label{sec:introduction}

Adaptive agents, for example humans, tend to explore new policies to try to outperform opponents in competitive settings. One possible approach is to reason about the unrevealed mental content of the other agents and act accordingly to their state of mind. This method of reasoning about the others' mental state is called \textit{Theory of Mind} \citep{premack1978does}, which has been found to be important for social competence \citep{liddle2006higher} and other settings like negotiation \citep{de2017negotiating} and the \textit{Mod Game} \citep{veltman2019training}, which will be further discussed in this paper.

Theory of Mind (ToM) is the ability to understand that other people have their own thoughts, beliefs, and intentions that may be different from one's own. It is a key component of social cognition, which is the ability to understand and interpret the actions and intentions of others. Theory of Mind is a crucial skill for successful social interactions, as it allows individuals to take the perspective of others, predict their behavior, and respond appropriately. Theory of Mind is a skill that develops over time, and is typically fully developed by the age of 5 \citep{call2008does}. However, it may not be limited to humans, and has been observed in other species such as chimpanzees \citep{premack1978does}.

In competitive settings, Theory of Mind can be an important factor in predicting and understanding the behavior of one's opponents. This is also supported by the Machiavellian Intelligence Hypothesis, which states that the evolution of intelligence is driven by the need to outwit others, as social success (achieved mainly using methods such as deception and manipulation, which requires a higher level of Theory of Mind) leads to reproductive success \citep{gavrilets2006dynamics}, which is one of the most driving factors of evolution. For example, individuals with a well-developed Theory of Mind may be able to anticipate their opponents' moves in a game of Rock-Paper-Scissors, giving them an advantage over their opponents who use Theory of Mind to a lesser extent\citep{de2013much}. Additionally, individuals with a strong Theory of Mind may be better able to interpret their opponent's non-verbal cues and respond accordingly. 

Overall, Theory of Mind is an important aspect of social cognition that can play a significant role in competitive settings not only for humans but for agents as well. Theory of Mind can also be applied recursively, in the sense that an agent can reason about the Theory of Mind of another agent, and so on. This is sometimes mentioned as \textit{Higher Order Theory of Mind}, where each consecutive order adds another layer of reasoning on top of the previous one. In \cite{de2013much} it is shown that in many different competitive settings, higher orders of Theory of Mind have a competitive advantage over lower levels, but it should be noted that the competitive advantage between levels decreases as the levels go up and tests were done for only up to the fourth level Theory of Mind.

The Mod Game, as described in \cite{frey2013cyclic} and \cite{veltman2019training}, is a general sum n-player game in which the participants simultaneously choose a number $k$ in the range $\{0, …, m\}$ where $m, n > 0$. In a general sum game, the total value of the payoffs for all players does not have to add up to a fixed number. After a player chooses number k, they gain one point for each other player that chose $k - 1$ as their number. For example if a player chose 8, for each opponent that chose 7 they gain one point. One notable exception is the case of 0, in which players gain one point for each player that chose $m$. The Mod Game is similar to other competitive games used in Theory of Mind settings, such as Rock, Paper and Scissors in the sense that actions have dominance over another (choosing 17 ``dominates/beats" choosing 16 and Rock dominates/beats Scissors). However, the Mod Game is different in the sense that it is a general sum game, and the payoffs for each player do not have to add up to a fixed number.

We will simulate a version of the Mod Game with signaling. In this version of the game, there will be two types of agents: signalers and receivers. The signaling agents will use their beliefs and intentions (which are updated every iteration using their chosen $\{Signal, Action\}$ pair and the choices of other agents in that iteration) to choose a $\{Signal, Action\}$ pair. They will then signal their chosen Signal to the receiving agents. The receiving agents will receive the signals and use their beliefs and intentions (which are updated every iteration using their chosen action, the signals received, and the choices of other agents in that iteration) to choose an Action. Finally, both the signaling agents and the receiving agents will act, and their choices will be registered for that turn. The scores of the agents and their beliefs and intentions will be updated accordingly.

The signaling in this scenario is a perfect example of \textit{Cheap Talk}, which is a form of communication that is cost-free and does not affect the payoffs of the agents directly. In \cite{farrell1996cheap}, it is shown that even if there is a small incentive to lie in a setting where cheap talk is possible, cheap talk can convey information and alter the Nash equilibrium. In the study it is also shown that in many discussed situations when the signals are not ``self-committing", they do not alter the Nash equilibrium. The study has arguments for both sides of the spectrum, but it also suggests that in settings like Prisoner's Dilemma \citep{kuhn1997prisoner} most authors and scientists emphasize on the fact that the prisoners cannot communicate, as with even semi-credible cheap talk, the prisoners can change the overall game, which would lead to a new Nash equilibrium. This can be exemplified by Prisoner 1 saying "If you snitch on me, I'll kill you" to Prisoner 2 and being believed. Even though that statement from Prisoner 1 is not self-committing, \citep{osborne1994course}, Prisoner 2 would have to consider that in their decision making process.

This version of the Mod Game with signaling was chosen to be investigated in order to explore the effect of theory of mind on the emergent strategies/behaviour previously discussed in this setting \citep{veltman2019training}, as well as whether the addition of signaling leads to increased cooperation, increased deception particularly by the receiving agents, or any other notable differences in agent behaviour. Specifically, we will examine how the inclusion of signaling affects the choices and strategies used by the agents, and whether it leads to any significant changes in their behavior. By studying the strategies used by the signaling agents to choose their {Signal, Action} pairs and the strategies used by the receiving agents to choose their actions based on the previous actions and signals received, we hope to shed light on our research question and better understand the role of theory of mind in this version of the Mod Game with signaling. Another research goal is to see if the addition of cost-free signaling leads to significant behavioral differences compared to the agent behaviour and scores on the regular simulation.  