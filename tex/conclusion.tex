\section{Conclusion}\label{sec:conclusion}

% 1. Recap of the Research Question and Objectives:
In this study, we have tried explore the intricate interplay between Theory of Mind (ToM) and strategic decision-making in competitive settings, in a very specific setting we have formulated. Our primary research question centered around the effects of ToM on agent behavior, with a specific focus on our setting:  the Mod Game with signaling. By simulating a total of 16 simulations with different agent compositions, we hoped to shed light on the effect of communication in competitive settings like this where there's traditionally no communication. Even though previous studies with the non-signaling version of the Mod Game exist, we developed an in-house simulation for both the signaling and non-signaling simulations inside the same framework for easier comparison and validity of results.

% 2.Summarize key findings 
In the regular simulation, we've found out that the agents with higher ToM agents were usually out-performing lower ToM agents, employing greedy strategies supported by the Machiavellian Intelligence Hypothesis. This was reflected in the result trends, where the score of $\mathbf{N}$ Order ToM agents were more often that not directly correlated with the number of $\mathbf{N-1}$ Order ToM agents. Only in 1 out of 7 simulations the higher order ToM dominance was not observed, with the first order agents performing better than second order agents, which strenghthens the hypothesis that higher order ToM agents are more likely to outperform lower order ToM agents, in the absence of communication.

In the signaling simulation, we've seen the same trend order-wise, higher order beats the lower order. However in each order, we had two groups of agents: one signaling and one receiving. When these sub-groups were inspected individually, we've seen an unexpected trend, with the first order receiving agents being the highest performers in 5 out of 9 simulations. This was surprising for two reasons: 1. There is a clear trend of lower order advantage over an higher order and 2. we expected the signaling agents to perform better than the receiving agents, because we thought they would employ deceptive strategies with signaling first and deciding second. However, the results showed that the receiving agents were the ones who were advantageous and getting more scores. While this could be attributed to them simply employing a greedy strategy over signaling agents, there are results that say otherwise. 

The score differences between orders, which is the metric we used to measure cooperation on an abstract level, is visibly and significanly lower the signaling simulation than the regular one. This metric shows the "inequality" between the score distribution between orders and essentially is a measure of how harsh higher orders of ToM affect the scores in competitive settings such as this. We believe, while it may simply be harder to effectively employ greedy strategies due to the complexity of the simulation compared to the regular one, that the addition of cost-free communication has a significant effect on the scores and the cooperation between agents.

% 3. Impact of Signaling:
Signaling, which is basically cost-free communication on a 1-dimensional level, was added to this scenario to see if agents would be more willing to cooperate given that it would be easier to do so. However, we've uncovered some interesting results that it is not as linear as we have hypothesized beforehand, but leads to way more complex emergent behaviour. The difference in score differences indicate that relatively more cooperative strategies are used, but not clear-cut cooperative or greedy like we have thought. This is an exciting result, as we see clear behavioral changes in agents when communication is introduced, and we believe that this is a very interesting area to explore further. This also supports the claim in \cite{farrell1996cheap} where it is stated that the addition of communication would drastically alter the Nash Equilibria and/or the behaviour of agents in many competitive settings, which we have observed in our simulation. Due to this, we also believe that cost-free communication like this could be introduced to many different competitive settings in future work, to see how it would affect the behaviour of agents in those settings. Especially, the addition of signaling in natural selection-like scenarios could be very interesting to observe, as it could lead to more cooperative strategies being employed by agents, which could lead to more interesting results overall.

% 4. Limitations and Future Work:
One possible limitation of this study is the process order, where the signaling and receiving process is synchronous and in order rather than asycnhronous and free flowing like real life human communication. While it is not a clear limitation per se, as in many human communication scenarios there are rules and a structure, it might be influencing the current behaviour of agents. That's why a follow-up study with asynchronous communication could also be worthwile to see if the results would be different. 

Another limitation is the fact that we have only used one type of communication, which is signaling. While it is a very simple form of communication, it is still communication and it could be interesting to see how other forms of communication would affect the results. For example, if we were to introduce a cost to communication, would the results be different? Or if we were to introduce a more complex form of communication, like a 2-dimensional one, would the results be different? These are all interesting questions that could be explored in future work.

% 5. Conclusion and final thoughts:
In this study, we have simualated the Mod Game with and without cost-free communication played by 300 Theory of Mind agents. We have found out that the addition of communication has a significant effect on the scores and the cooperation between agents, and changes the dynamic and behaviour of agents on a noticeable level. The agents had a more even distribution of scores in the simulation with communication, leading us to believe that the agents employ a relatively more cooperative strategy. While the employed experimental methodology may have been limited in some aspects, we believe that the results merit further exploration of the effects of communication in competitive settings like this.
