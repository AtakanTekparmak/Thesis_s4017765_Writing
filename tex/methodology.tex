\section{Methodology}\label{sec:methodology}

The main experiment methodology of this paper will involve simulating the Mod Game with and without signaling, using artificial agents. Different initial populations of agents will be used in the simulation, with varying levels of Theory of Mind as described in \cite{premack1978does} and \cite{liddle2006higher}. The behavior of the agents will be observed and analyzed through and across the simulations to investigate the effect of different order configurations and the addition of signaling on the behavior of Theory of Mind agents in the Mod Game, as described in \cite{veltman2019training} and \cite{frey2013cyclic}.

To begin, a computer program will be created to simulate the Mod Game with and without signaling. The program will be written in the Python \citep{van1995python} programming language, following the implementation of the Mod Game as described in \cite{veltman2019training}. The program will be designed to simulate the game with any given number of agents (initial population configurations), and will include functionality for agents to signal and receive signals in the signaling simulation, on top of the general ability of being able to update their beliefs and intentions at every iteration/round that is present in the non-signaling version.

The simulation will be run multiple times, each with a different initial population of agents. The initial populations will be varied in terms of the levels of Theory of Mind of the agents, as described in \cite{de2013much}. Each initial configuration of agents will be simulated 10 times and their results will be aggregated to obtain more sound results. For this version of the Mod Game, we chose $m = 22$.

\subsection{The Regular Simulation}

In the regular simulation, which is a basic n-player version of the Mod Game with Theory of Mind agents, three different processes are performed in each iteration/round of the simulation. These processes are:
\begin{enumerate}
    \item The agents \textbf{decide} based on their beliefs,
    \item The agent scores are updated, and
    \item The agents beliefs are \textbf{update}d.
\end{enumerate}

The \textbf{decide} and \textbf{update} processes are performed differently for each level of Theory of Mind. However, there are similarities between the processes, as per the requirement for a higher level Theory of Mind: accurately modeling the lower levels of Theory of Mind. How each order of Theory of Mind agent performs these processes is described below.

\subsubsection{Zero Order Theory of Mind Agent}

The Zero Order Theory of Mind agent is the simplest agent in the simulation. It does not have any Theory of Mind, and tries to model the other agents as simple as possible through a 1-dimensional belief vector that represents the agent's belief in most common action chosen by the other agents. It simply uses this vector to choose an action and update its beliefs. The \textbf{decide} function can be described as follows:

Let $\mathbf{b} = [b_0, b_1, \ldots, b_{22}]$ be the belief vector, where $b_i$ represents the belief in any given opponent choosing action $i$. The function iterates through the beliefs and finds the index $i^*$ corresponding to the highest belief:

\[
i^* = \arg\max_i b_i
\]

To incorporate exploration, a random action is chosen with probability $\epsilon$ using the epsilon-greedy strategy. The function employs the check\_epsilon function to determine if a random choice should be made. The zero order decision $d_0$ is then given by:

\label{eq:zero-order-decide}
\[
\begin{aligned}
d_0 =
\begin{cases}
\text{{random\_choice()}}, & \text{{if }} \text{{check\_epsilon()}} \\
i^* + 1 \mod 23, & \text{{otherwise}}
\end{cases}
\end{aligned}
\]

Here, $\text{{random\_choice()}}$ generates a random action, and $i^* + 1 \mod 23$ ensures that the action is within the range of 0 to 22.

The \textbf{update} function of the Zero order Theory of Mind Agent updates the belief vector, using the actions of other agents $\mathbf{a} = [a_0, a_1, \ldots, a_{n-1}]$ where $n$ is the number of agents, as follows: 

\label{eq:zero-order-update}
Let $\mathbf{b} = [b_0, b_1, \ldots, b_{22}]$ be the belief vector, where $b_i$ represents the belief in action $i$. First, the function computes the accumulated sum of beliefs, denoted as $A$, using the equation:

\[
A = \sum_{i=0}^{22} b_i.
\]

Next, it calculates the learning rate $\alpha$ as:

\[
\alpha = \frac{{1 - \text{{LEARNING\_SPEED}}}}{{A}}.
\]

The learning rate $\alpha$ represents the proportion of belief update for each action. A higher accumulated sum of beliefs leads to a smaller learning rate, promoting stability and slower belief adjustments. Finally, the function updates each belief $b_i$ by multiplying it with the learning rate $\alpha$, except for the belief corresponding to the chosen action. The updated belief vector is given by:

\[
\begin{aligned}
b_i & = b_i \times \alpha, \quad i = 0, 1, \ldots, 22, \quad\\
b_{a_{j}} & = b_{a_{j}} + \text{{LEARNING\_SPEED}},
\end{aligned}
\]

where $a_j$ represents an action from the list $\mathbf{a}$ that represents the list of actions of the other agents in that round. This process is repeated for each action in the actions list. One notable feature of this update equation is that it is performed sequentially according to the initialization order of the agents, so the order of the agents can affect the beliefs of the agents. The update equation ensures that the belief corresponding to the chosen action is increased by the predefined learning speed $\text{{LEARNING\_SPEED}}$, while the other beliefs are scaled down proportionally.

\subsubsection{First Order Theory of Mind Agent}

The First Order Theory of Mind agent is a more complex agent than the Zero Order Theory of Mind agent. It models the other agents in the simulation as Zero Order Theory of Mind agents, uses that model in deciding and updating. They, like the zero order agents, have a fixed order Theory of Mind

The \textbf{decide} function of the First Order Theory of Mind Agent models the decision-making process of a lower-order agent (Zero order Theory of Mind Agent) and acts greedily based on that model. First, the function invokes the \textbf{decide} function of the lower-order agent, which determines the action the lower-order agent would choose based on its beliefs and decision-making strategy. Next, the function makes its decision $\mathbf{d_1}$ by applying a greedy strategy based on the lower-order decision \hyperref[eq:zero-order-decide]{$\mathbf{d_0}$}:

\[
d_1 = (d_0 + 1) \mod 23.
\]

Because the First Order agent using a Zero Order agent to decide, the \textbf{update} procedure is just updating that Zero Order agent, as described in the previous subsection. It should be noted that unlike \cite{veltman2019training} first order agents are fixed to the first order of thinking, and cannot reason in a zero order way.

\subsubsection{Second Level Theory of Mind Agent}

The Second Order Theory of Mind agent is very similar to a First Order Theory of Mind agent, with the main difference being that it has another dimension of reasoning, namely its beliefs on which order of Theory of Mind is more present in the population, Zero Order or First Order. It then uses this belief to decide which lower order agent to model and act upon. 

The \textbf{decide} function of the Second Order Theory of Mind Agent implements a greedy decision-making process based on models of the other agents that it has. First, the function selects an order that it believes to be more present in the equation. Let $\mathbf{b_o} = [b_{o_{1}}, b_{o_{2}}]$ be the order belief vector. Then, the intermediary order-decision $o^*$ is:

\[
\begin{aligned}
\text{{$o^*$}} =
\begin{cases}
0, & \text{{if }} b_{o_{1}} > b_{o_{2}}\\
1, & \text{{otherwise}}
\end{cases}
\end{aligned}
\]

In the order decision procedure there is epsilon-based stochasticity as well, the final order decision $o$ is selected with:

\[
\begin{aligned}
\text{{$o$}} =
\begin{cases}
\text{{random\_choice()}}, & \text{{if }} \text{{check\_epsilon()}} \\
o^*, & \text{{otherwise}}
\end{cases}
\end{aligned}
\]

where \textit{random\_choice()} randomly chooses between 0 and 1. The agent then invokes the \textbf{decide} function of the corresponding lower-order agent to get the decision $d_l$ of the lower-order agent. Then, the second order decision $\mathbf{d_2}$  is calculated using the equation:

\label{eq:second-order-decide}
\[
d_2 = (d_l + 1) \mod 23.
\]


The \textbf{update} function of the Second Order Theory of Mind Agent updates the beliefs of the zero order, first order, and second order agents based on the chosen action. First, the function retrieves the decisions made by the first order agents using their \textbf{decide} functions. The higher order decisions $\mathbf{d_{h_0}}$ and $\mathbf{d_{h_1}}$ are calculated using the equations:

\[
d_{h_0} = d_1, \quad d_{h_1} = (d_1 + 1) \mod 23.
\]

Next, the function updates the beliefs of the zero order and first order agents by invoking their respective \textbf{update} functions. The function then updates the order beliefs vector based on the chosen action. If the chosen action matches either the zero order higher decision $\mathbf{d_{h_0}}$ or the first order higher decision $\mathbf{d_{h_1}}$, the order beliefs $\mathbf{b_o}$ are updated as follows:

Let $a$ be the chosen action. The sum of the order beliefs is calculated as $s = b_{o_0} + b_{o_1}$, and the value of $s$ is used to calculate an accumulation factor $f$:

\[
f = \frac{{1.0 - \text{{LEARNING\_SPEED}}}}{s}.
\]

The belief values in the order beliefs vector are scaled down by multiplying each belief $b_{o_i}$ by the accumulation factor $f$. If $a$ matches the zero order higher decision $\mathbf{d_{h_0}}$, the belief in the majority of zero order agents is updated by:

\[
b_{o_0} = b_{o_0} + \text{{LEARNING\_SPEED}}.
\]

Otherwise, if $a$ matches the first order higher decision $\mathbf{d_{h_1}}$, the belief in the majority of first order agents is updated by:

\[
    b_{o_1} = b_{o_1} + \text{{LEARNING\_SPEED}}.
\]

\subsubsection{Simulation}

The non-signaling simulation will have a total of 7 different initial population configurations to be simulated, which are listed in \hyperref[tab:reg-population-table]{Table 2.1}. These population configurations are chosen specifically because for each ToM order it investigates the effects of:
\begin{itemize}
    \item The Equal Population: $\{33\%,33\%,33\%\}$,
    \item The Surplus of an Order: $\{50\%,25\%,25\%\}$ and 
    \item The Deficiency in an Order: $\{20\%,40\%,40\%\}$
\end{itemize}

\begin{table}[h]
\centering
\begin{tabular}{ccc}
\hline
\multicolumn{1}{|c|}{\begin{tabular}[c]{@{}c@{}}0\\ Order\end{tabular}} & \multicolumn{1}{c|}{\begin{tabular}[c]{@{}c@{}}1st\\ Order\end{tabular}} & \multicolumn{1}{c|}{\begin{tabular}[c]{@{}c@{}}2nd\\ Order\end{tabular}} \\ \hline
\multicolumn{1}{c|}{33}                                                 & \multicolumn{1}{c|}{33}                                                  & 33                                                                       \\
\multicolumn{1}{c|}{50}                                                 & \multicolumn{1}{c|}{25}                                                  & 25                                                                       \\
\multicolumn{1}{c|}{20}                                                 & \multicolumn{1}{c|}{40}                                                  & 40                                                                       \\
\multicolumn{1}{c|}{25}                                                 & \multicolumn{1}{c|}{50}                                                  & 25                                                                       \\
\multicolumn{1}{c|}{40}                                                 & \multicolumn{1}{c|}{20}                                                  & 40                                                                       \\
\multicolumn{1}{c|}{25}                                                 & \multicolumn{1}{c|}{25}                                                  & 50                                                                       \\
\multicolumn{1}{c|}{40}                                                 & \multicolumn{1}{c|}{40}                                                  & 20                                                                       \\
   & &                
\end{tabular}
\caption{The initial population configurations listed by how much each order of ToM is present in the population percentage-wise}
\label{tab:reg-population-table}
\end{table}

\subsection{The Signaling Simulation}

In the signaling simulation, the agents are divided into two groups: the signaling agents and the receiving agents. The signaling agents can signal any action (a number in the range $[0,22]$) to the receiving agents, on top of the ability to decide. The receiving agents can process transmitted signals from the signaling agents before deciding. There are five different processes performed in each round. In order, they are:

\begin{enumerate}
    \item The signaling agents decide on a signal and transmit it
    \item The receiving agents receive the signal and process it
    \item All agents decide on an action
    \item The agent scores are updated  
    \item The agent beliefs are updated
\end{enumerate}

Even though not mentioned in the itemized list above, the decision and update processes for signaling and receiving agents are different. They do resemble in certain aspects as similar learning algorithms were used for consistency, but the differences are significant enough to warrant a separate explanation, which will be done in the following subsections.

\subsubsection{Signaling Agents}

All signaling agents employ a 2-dimensional beliefs array of the shape (23,23), in which the first dimension represents and holds information for signals, and the second dimension represents and holds information for actions given the signal. In the decision process for all signaling agents, the agent uses the sub-array of shape (23,1), indexed by the signal, to decide on an action. How each order of ToM deals with the signal, decision and update processes are explained in the following sections.

\subsubsubsection{2.2.1.1}{Zero Order Signaling Agent}

The zero order agent employs a simple greedy strategy for signaling and deciding. In both of them, the belief vector $\mathbf{b_{\zeta}}$ is used, where: 

\[
\mathbf{b_{\zeta}} = [[b_{\zeta_{0_{0}}}, \ldots, b_{\zeta_{0_{22}}}], \ldots, [b_{\zeta_{22_{0}}}, \ldots, b_{\zeta_{22_{22}}}]]
\]

The \textbf{signal} $\mathbf{\phi}$ and the \textbf{decision} $\mathbf{d_0}$, which is chosen by the agent using the signal $\mathbf{\phi}$ is given by: 

\begin{equation}
\label{eq:zero-order-signal}
\begin{aligned}
    \phi = \arg\max_s \sum_{i=0}^{22} b_{\zeta_{s_i}}, \quad i = 0, 1, \ldots, 22. \\
    d_0 = \arg\max_d b_{\zeta_{\phi_d}}, \quad d = 0, 1, \ldots, 22.
\end{aligned}
\end{equation}

For each action $\mathbf{a}$ taken by the other agents during that epoch, the agent employs a process to decide whether or not the action should be considered in the update procedure, outlined as follows: 

\begin{equation}
\text{{update}}(a) \rightarrow \begin{cases}
\text{{yes}}, & \text{{if }} d_0 = (a + 1) \mod 23 \\
\text{{no}}, & \text{{otherwise}}
\end{cases}
\end{equation}

After that, the update process is the same as the \hyperref[eq:zero-order-update]{Non-signaling Zero Order Agent}, with the only difference being instead of $\mathbf{b}$, the sub-vector $\mathbf{b_{s^0}}$ is used.

\subsubsubsection{2.2.1.2}{First Order Signaling Agent}

The first order agent employs a similar but slightly different strategy for signaling and deciding. In both of them, the same belief vector $\mathbf{b}$ is used. The agent also has a zero order agent model $\mathbf{m^0}$, whose \textbf{signal} $\mathbf{\phi}$ (calculations given in \hyperref[eq:zero-order-signal]{Equation 2.1}) is used in the  signaling process. The \textbf{signal} $\mathbf{s^1}$ and the \textbf{decision} $\mathbf{d_1}$ chosen by the agent is given by:

\begin{equation*}
\begin{aligned}
    s^1 = \begin{cases}
        \text{{random\_choice()}}, & \text{{if }} \text{{check\_epsilon()}} \\
        s^0, & \text{{otherwise}}
    \end{cases} \\
    d_1 = \arg\max_d b_{{s^1}_d}, \quad d = 0, 1, \ldots, 22. \qquad
\end{aligned}
\end{equation*}

The \textbf{update} process is the same as the \hyperref[eq:zero-order-update]{Non-signaling Zero Order Agent}, with the only difference being instead of $\mathbf{b}$, the sub-vector $\mathbf{b_{s^1}}$ is used.

\subsubsubsection{2.2.1.2}{Second Order Signaling Agent}

The second order signaling agent is a lot like a regular second order agent, with the lower order signaling agent models $\mathbf{m^0}$ and $\mathbf{m^1}$, and the order beliefs $\mathbf{b_o}$. The \textbf{decide} and \textbf{update} processes are the same as the \hyperref[eq:second-order-decide]{Non-signaling Second Order Agent}, with the only difference being the extra \textbf{signal} $\mathbf{s^2}$ chosen by the agent, given by the same algorithm for \textbf{Non-signaling Second Order decide}, but without the modulo operation on the lower order signal $\mathbf{s_l}$.

\subsubsection{Receiving Agents}

The receiving agents make use of two different sets of beliefs, one 1-dimensional and one 2-dimensional. The 1-dimensional beliefs array of the shape (23,1) is used to process the incoming signals, and the 2-dimensional beliefs array of the shape (23,23) is used to decide on an action according to the current beliefs formulated through received signals. How each order of ToM deals with the processing the signal, decision and update processes are explained in the following subsections.

\subsubsubsection{2.2.2.1}{Zero Order Receiving Agent}

The zero order receiving agent employs a simple greedy strategy for processing the signal and deciding. In the \textbf{process\_signal} process, the agent uses the the belief vector $\mathbf{b_s}$, where:

\[
\mathbf{b_s} = [b_{0}, b_{1}, \ldots, b_{22}]
\]

on this vector, which is re-initialised every round, the agent performs the same \textbf{update} process as the \hyperref[eq:zero-order-update]{Non-signaling Zero Order Agent}, with the only difference being instead of $\mathbf{b}$, the vector $\mathbf{b_s}$ is used. The vector is re-initialised because it represents the agent's beliefs about the most dominant signal in the current round, and the agent needs to re-evaluate this belief every round. The learning rate for the update operations of this vector is 4 times the simulation-wide learning rate $\alpha = 0.1$.
For the decide and update processes the connected beliefs vector $\mathbf{b_c}$ is used, where:

\[
\mathbf{b_c} = [[b_{c_{0_{0}}}, b_{c_{0_{1}}}, \ldots, b_{c_{0_{22}}}], \ldots, [b_{c_{22_{0}}}, b_{c_{22_{1}}}, \ldots, b_{c_{22_{22}}}]]
\]

The \textbf{decide} process is given by, where $\psi$ represents the zero order signal belief for the round:

\begin{equation*}
\begin{aligned}
    \psi = \arg\max_s b_{s_i}, \quad i = 0, 1, \ldots, 22. \\
    d_0 = \arg\max_d b_{c_{{\psi}_d}}, \quad d = 0, 1, \ldots, 22.
\end{aligned}
\end{equation*}

The \textbf{update} process is the same as the \hyperref[eq:zero-order-update]{Non-signaling Zero Order Agent}, with the difference being that the connected beliefs vector $\mathbf{b_{c_\psi}}$ is used instead of $\mathbf{b}$, and the following calculation is used:

\begin{equation*}
    b_{c_{\psi_{a_j}}} = b_{c_{\psi_{a_j}}} + LEARNING\_SPEED
\end{equation*}

where $a_j$ represents an action from the actions list for that round.

\subsubsubsection{2.2.2.2}{First Order Receiving Agent}

The first order receiving agent is a lot like the non-signalling first order agent, in the sense that it simply uses a model of a zero order agent to operate, with the sole addition of having a \textbf{process\_signal} process, where the \textbf{process\_signal} process of the zero order agent is called. This agent is essentially a superset of the non-signaling first order agent, with the extra ability to process signals.

\subsubsubsection{2.2.2.3}{Second Order Receiving Agent}

The second order receiving agent is also almost identical the non-signaling second order agent, with only the extra \textbf{process\_signal} process, in which the \textbf{process\_signal} process of the zero and first order agents are called. This agent is also essentially a superset of the non-signaling second order agent, with the extra ability to process signals.

\subsubsection{Simulation}

The signaling simulation will have a total of 11 different initial population configurations to be simulated. These population configurations are chosen specifically because for each ToM order it again investigates the effects listed for the signaling simulation, but also the effects of scenarios in which there are more signaling agents than receiving agents, and vice versa, based on the default order population of the simulation (i.e. the first row of \hyperref[tab:reg-population-table]{Table 2.1}). The ratio of difference for the scenarios with more signaling agents than receiving agents is 30:70 and for the scenarios with more receiving agents it's 70:30. The population configurations are like described below:

\begin{itemize}
    \item Order-wise, the initial 7 scenarios from \hyperref[tab:reg-population-table]{Table 2.1} are exactly the same, with each order being divided into half for signaling and receiving agents.
    \item The other 4 scenarios are with the order configuration of $\{33,33,33\}$ but with the signaling agents being 70\% and 90\% of the population and the receiving agents being 30\% and 10\% of the population, and vice versa.
\end{itemize}


Throughout the simulation, the behavior of the agents will be observed and recorded. This data will include the signals sent by the signaling agents, the actions chosen by all agents, and the payoffs received by each agent. The data will be analyzed, for each simulation individually but also inter-simulation, to determine how the inclusion of signaling in the Mod Game affects the behavior of Theory of Mind agents. The results will be compared to the findings of \cite{de2013much} and \cite{veltman2019training} to evaluate the effect on signaling on emergent agent behaviour.