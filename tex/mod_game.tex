\section{The Mod Game}\label{sec:mod_game}

The Mod Game, as described in \cite{frey2013cyclic} and \cite{veltman2019training}, is a general sum n-player game in which the participants simultaneously choose a number $k$ in the range $\{0, …, m\}$ where $m, n > 0$. In a general sum game, the total value of the payoffs for all players does not have to add up to a fixed number. After players choose their numbers, for each player that chose $k - 1$ as their number, players gain one point. For example if a player chose 8, for each opponent that chose 7 they gain one point. One notable exception is the case of 0, in which players gain one point for each player that chose $m$. The Mod Game is similar to other competitive games used in Theory of Mind settings, such as Rock, Paper and Scissors in the sense that actions have dominance over another (choosing 17 ``dominates/beats" choosing 16 and Rock dominates/beats Scissors). However, the Mod Game is different in the sense that it is a general sum game, and the payoffs for each player do not have to add up to a fixed number.

We will simulate a version of the Mod Game with signaling. In this version of the game, there will be two types of agents: signalers and receivers. The signaling agents will use their beliefs and intentions (which are updated every iteration using their chosen $\{Signal, Action\}$ pair and the choices of other agents in that iteration) to choose a $\{Signal, Action\}$ pair. They will then signal their chosen Signal to the receiving agents. The receiving agents will receive the signals and use their beliefs and intentions (which are updated every iteration using their chosen action, the signals received, and the choices of other agents in that iteration) to choose an Action. Finally, both the signaling agents and the receiving agents will act, and their choices will be registered for that turn. The scores of the agents and their beliefs and intentions will be updated accordingly.

The signaling in this scenario is a perfect example of \textit{Cheap Talk}, which is a form of communication that is cost-free and does not affect the payoffs of the agents directly. In \cite{farrell1996cheap}, it is shown that even if there is a small incentive to lie in a setting where cheap talk is possible, cheap talk can convey information and alter the Nash equilibrium. In the study it is also shown that in many discussed situations when the signals are not "self-committing", they do not alter the Nash equilibrium. The study has arguments for both sides of the spectrum, but it also suggests that in settings like Prisoner's Dilemma \citep{kuhn1997prisoner} most authors and scientists emphasize on the fact that the prisoners cannot communicate, as with even semi-credible cheap talk, the prisoners can change the overall game, which would lead to a new Nash equilibrium. This can be exemplified by Prisoner 1 saying "If you snitch on me, I'll kill you" to Prisoner 2 and being believed. Even though that statement from Prisoner 1 is not self-committing, \citep{osborne1994course}, Prisoner 2 would have to consider that in their decision making process.

This version of the Mod Game with signaling was chosen to be investigated in order to explore the effect of theory of mind on the emergent strategies/behaviour previously discussed in this setting \citep{veltman2019training}, as well as whether the addition of signaling leads to increased cooperation, increased deception particularly by the receiving agents, or any other notable differences in agent behaviour. Specifically, we will examine how the inclusion of signaling affects the choices and strategies used by the agents, and whether it leads to any significant changes in their behavior. By studying the strategies used by the signaling agents to choose their {Signal, Action} pairs and the strategies used by the receiving agents to choose their actions based on the previous actions and signals received, we hope to shed light on our research question and better understand the role of theory of mind in this version of the Mod Game with signaling. Another research goal is to see if the addition of cost-free signaling leads to significant behavioral differences compared to the agent behaviour and scores on the regular simulation.  