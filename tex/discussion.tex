\section{Discussion}\label{sec:discussion}

\subsection{The Regular Simulation}

In the regular simulation, it is seen that including the default equal population, in 6 out of 7 tested scenarios, agents perform better with a higher order ToM with the first order agents performing better than zero order agents and second order agents performing better than the first order agents. 

The only exception is the $150_{0}$/$75_{1}$/$75_{2}$ scenario, in which there are $50\%$ more zero order agents than the default equal population. In this scenario, the first order agents perform better than the second order agents but also this is the scenario with the highest total score across all three orders of ToM. This is because while second order agents can adapt to a zero order dominant population, first order agents are ready for it and benefit from it from the beginning. This is evident in all the other scenarios too, as the first order score is directy correlated with the number of zero order agents. 

The second order agents are not as sensitive to population changes, but they also prefer more zero order agents than more first order agents, with them having higher AOAS in scenarios with more zero order agents.

These results indicate that without signaling, unless there is a significant (around $50\%$) increase in the number of zero order agents, higher order agents, in general, perform better than lower order agents (with the highest order being 2). The average agent score (across all orders) for 300 agents and 1000 epochs is $\mathbf{63.115
}$ for the default equal population, and $\mathbf{78.52}$ for the $150_{0}$/$75_{1}$/$75_{2}$ scenario, which has the highest average.

