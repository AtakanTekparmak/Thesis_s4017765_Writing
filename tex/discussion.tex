\section{Discussion}\label{sec:discussion}

\subsection{The Regular Simulation}

In the regular simulation, it is seen that including the default equal population, in 6 out of 7 tested scenarios, agents perform better with a higher order ToM with the first order agents performing better than zero order agents and second order agents performing better than the first order agents. 

The only exception is the $150_{0}$/$75_{1}$/$75_{2}$ scenario, in which there are $50\%$ more zero order agents than the default equal population. In this scenario, the first order agents perform better than the second order agents but also this is the scenario with the highest total score across all three orders of ToM. This is because while second order agents can adapt to a zero order dominant population, first order agents are ready for it and benefit from it from the beginning. This is evident in all the other scenarios too, as the first order score is directy correlated with the number of zero order agents. 

The second order agents are not as sensitive to population changes, but they also prefer more zero order agents than more first order agents, with them having higher AOAS in scenarios with more zero order agents.

These results indicate that without signaling, unless there is a significant (around $50\%$) increase in the number of zero order agents, higher order agents, in general, perform better than lower order agents (with the highest order being 2). The average agent score (across all orders) for 300 agents and 1000 epochs is $\mathbf{63.115
}$ for the default equal population, and $\mathbf{78.517}$ for the $150_{0}$/$75_{1}$/$75_{2}$ scenario, which has the highest average.

\subsection{The Signaling Simulation}

In the signaling simulation, the picture is not too different from the regular simulation if analysed from an order-wise perspective. In all of the scenarios, the first order agents collectively perform better than zero order agents, and second order agents collectively perform better than first order agents, with no exceptions. This is expected, as also seen in the regular simulation and in previous work, higher order agents performe better than lower order agents, with the highest order being 2.

The main difference between the regular and signaling simulations is the score differences. In the regular simulation, in all scenarios, at least one of the score differences (either $\mathbf{sd^0_1}$ or $\mathbf{sd^1_2}$) is higher than $\mathbf{48}$ whereas in the signaling simulation, in 8 out of 9 scenarios, both score differences are lower than $\mathbf{48}$. The only exception is the $60_{s_{0}}/60_{s_{1}}/30_{s_{2}}/60_{r_{0}}/60_{r_{1}}/30_{r_{2}}$ scenario where $\mathbf{sd^0_1 = 48.621}$. This indicates a more balanced distribution of scores between the orders of ToM in the signaling simulation, from a higher, order-wise perspective.