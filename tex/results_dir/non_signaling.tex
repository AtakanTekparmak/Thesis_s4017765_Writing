\subsubsection{Default Equal Population}

The default equal population consisted of 100 agents for each order, with total equality between populations with different orders of Theory of Mind. The end results are shown in \hyperref[table:non-sig-default-equal-simple]{Table 3.1}, in which the \hyperref[eq:aoas]{AOAS} is shown for each order, with their respective standard deviations. 

\begin{table}[h]
\centering
\begin{tabular}{|c|c|c|}
\hline
Order & AOAS & Std Dev \\ \hline
0     & 2.475 & 0.341   \\
1     & 83.448 & 0.814  \\
2     & 103.422 & 18.307 \\ \hline
\end{tabular}
\caption{The average order agent scores for each order in the the default equal population}
\label{table:non-sig-default-equal-simple}
\end{table}

The results show that the zero order agents, due to them having no Theory of Mind, have the lowest AOAS, $\mathbf{2.475}$, with a fairly low standard deviation. The first order agents have the 2nd highest AOAS, $\mathbf{83.448}$, with even lower standard deviation considering the higher score. The second order agents have the highest AOAS, $\mathbf{103.422}$, but with also the highest standard deviation, which is expected due to the higher variance in the decision-making of second order agents as they employ an extra set of beliefs about the predominant order of the population. The comparatively lower standard deviation of the first order agents might be due to the fact that they are kind of a generalization of the zero order decision-making process. These results indicate that in the current scenario with the default equal population, the second order agents are the most successful, followed by the first order agents, and then the zero order agents. The score difference between orders is $(\mathbf{sd^0_1 = 80.973}, \mathbf{sd^1_2 = 19.974})$, which indicates that the competitive advantage of the second order agents over the first order agents is not as high as the competitive advantage of the first order agents over the zero order agents. This score difference will also be used as baseline for comparison with the other initial population configurations.

\subsubsection{Zero Order Over and Under Abundance}

The zero order over-abundance scenario (percentages shown in the second row of \hyperref[tab:reg-population-table]{Table 2.1}) consisted of 150 zero order agents and 75 first and second order agents each. The zero order under-abundance scenario (percentages shown in the third row of \hyperref[tab:reg-population-table]{Table 2.1}) consisted of 60 zero order agents and 120 first and second order agents each. The end results are shown in \hyperref[table:non-sig-zero-order-simple]{Table 3.2}, which has two sub-tables in the same format as \hyperref[table:non-sig-default-equal-simple]{Table 3.1}.

\begin{table}[h]
\centering
\begin{tabular}{|c|c|c|}
\hline
\multicolumn{3}{|c|}{Over-abundance ($150_{0}$/$75_{1}$/$75_{2}$)} \\
\hline
Order & AOAS & Std Dev \\
\hline
0     & 2.453   & 0.352   \\
1     & 124.034 & 1.250   \\
2     & 109.063 & 13.018  \\
\hline
\end{tabular}
\qquad
\begin{tabular}{|c|c|c|}
\hline
\multicolumn{3}{|c|}{Under-abundance ($60_{0}$/$120_{1}$/$120_{2}$)} \\
\hline
Order & AOAS & Std Dev \\
\hline
0     & 2.473   & 0.378   \\
1     & 51.096  & 0.531   \\
2     & 99.207  & 43.861  \\
\hline
\end{tabular}
\caption{The average order agent scores for each order in the over-abundance and under-abundance scenarios, where the population is in the format $x_{0}$/$y_{1}$/$z_{2}$ for zero, first, and second order agents.}
\label{table:non-sig-zero-order-simple}
\end{table}

The results show that when there are more zero order agents, the first order agents have more agents to gain scores from and they do. The score difference between orders is $(\mathbf{sd^0_1 = 121.581}, \mathbf{sd^1_2 = -14.971})$, which indicates a drastic advantage shift to first order agents. Compared to the default equal population score difference, the differences between the first order and the zero order increased by $\mathbf{40.608}$, while the difference between the second order and the first order decreased by $\mathbf{33.945}$. This is in line with our expectations, as the first order agents are more efficient at gaining scores from zero order agents than second order agents. When there are less zero order agents, the first order agents have less agents to gain scores from and they do here too. The score difference between orders is $(\mathbf{sd^0_1 = 48.623}, \mathbf{sd^1_2 = 48.111})$, which indicates an even more advantageous scenario for second order agents. Compared to the default equal population score difference, the differences between the first order and the zero order decreased by $\mathbf{32.350}$, while the difference between the second order and the first order increased by $\mathbf{28.137}$. 

\subsubsection{First order Over and Under Abundance}

The first order over and under-abundance scenarios are given the population configurations $75_{0}$/$150_{1}$/$75_{2}$ and $120_{0}$/$60_{1}$/$120_{2}$. The results are shown in \hyperref[table:non-sig-first-order-simple]{Table 3.3}, in the same format as \hyperref[table:non-sig-zero-order-simple]{Table 3.2}.

\begin{table}[h]
\centering
\begin{tabular}{|c|c|c|}
\hline
\multicolumn{3}{|c|}{Over-abundance ($75_{0}$/$150_{1}$/$75_{2}$)} \\
\hline
Order & AOAS & Std Dev \\
\hline
0     & 2.473    & 0.379    \\
1     & 63.203   & 0.650    \\
2     & 107.780  & 41.247   \\
\hline
\end{tabular}
\qquad
\begin{tabular}{|c|c|c|}
\hline
\multicolumn{3}{|c|}{Under-abundance ($120_{0}$/$60_{1}$/$120_{2}$)} \\
\hline
Order & AOAS & Std Dev \\
\hline
0     & 2.458   & 0.341  \\
1     & 99.705  & 0.990  \\
2     & 101.938 & 3.497  \\
\hline
\end{tabular}
\caption{The average order agent scores for each order in the over-abundance and under-abundance scenarios for the first order.}
\label{table:non-sig-first-order-simple}
\end{table}

The results show that when there are more first order agents, the second order agents have more "prey" due to their ability to gain scores from first order agents. The score difference between orders is $(\mathbf{sd^0_1 = 60.730}, \mathbf{sd^1_2 = 44.577})$, which indicates a noticeable advantage shift to second order agents. Compared to the default equal population score difference, the differences between the first order and the zero order decreased by $\mathbf{20.243}$, while the difference between the second order and the first order increased by $\mathbf{24.604}$. This reflects the decreased opportunity for first order agents to gain scores due to less number of zero order agents and the increased opportunity for second order agents to gain scores due to more number of first order agents, which are more deterministic in their behaviour than the zero order agents (standard deviation percentage relative to the mean is $\mathbf{13.87\%})$ for the zero order agents and $\mathbf{0.99\%}$ for the first order agents). The relatively high standard deviation in the second order agent scores could be due to the fact that there are more first order agents that they have to model, which could lead to more variance due to the first order agents' stochastic behaviour.

When there are less first order agents, there are more zero order agents, which reflects the trend change favoring the first order, as shown in the \hyperref[table:non-sig-zero-order-simple]{Table 3.2} for the zero order over-abundance scenario, but in a less drastic manner. The score difference between orders is $(\mathbf{sd^0_1 = 97.247}, \mathbf{sd^1_2 = 2.233})$, which is almost a mean of the score difference for the default equal population $(\mathbf{sd^0_1 = 80.973}, \mathbf{sd^1_2 = 19.974})$ and the zero order over-abundance scenario ($\mathbf{sd^0_1 = 121.581}, \mathbf{sd^1_2 = -14.971}$). This shows that the trend is stable and more zero order agents mean more scores for the first order agents. In this case we have the lowest standard deviation across orders so far, with the second order agents having a notable standard deviation percentage relative to the mean of $\mathbf{3.43\%}$. This might be due to the fact that zero and second order agents are more deterministic in their behaviour in this scenario than first order agents.


\subsubsection{Second Order Over and Under Abundance}

The second order over and under-abundance scenarios are in the same format as the zero and first order over and under-abundance scenarios, but with the second order agents. The end results are shown in \hyperref[table:non-sig-second-order-simple]{Table 3.4}.

\begin{table}[h]
\centering
\begin{tabular}{|c|c|c|}
\hline
\multicolumn{3}{|c|}{Over-abundance ($75_{0}$/$75_{1}$/$150_{2}$)} \\
\hline
Order & AOAS & Std Dev \\
\hline
0     & 2.477   & 0.361    \\
1     & 63.268  & 0.582    \\
2     & 93.250  & 27.571   \\
\hline
\end{tabular}
\qquad
\begin{tabular}{|c|c|c|}
\hline
\multicolumn{3}{|c|}{Under-abundance ($120_{0}$/$120_{1}$/$60_{2}$)} \\
\hline
Order & AOAS & Std Dev \\
\hline
0     & 2.482   & 0.356   \\
1     & 99.689  & 0.979  \\
2     & 111.518 & 11.331 \\
\hline
\end{tabular}
\caption{The average order agent scores for each order in the over-abundance and under-abundance scenarios for the second order.}
\label{table:non-sig-second-order-simple}
\end{table}

The results show that when there are more second order agents, the overall score decreases with the first order agent scores taking a decent hit. The score difference between orders is $(\mathbf{sd^0_1 = 60.791}, \mathbf{sd^1_2 = 29.982})$. When there are less second order agents, both the first and second order agents have more scores. The score difference between orders is $(\mathbf{sd^0_1 = 97.207}, \mathbf{sd^1_2 = 11.829})$. One important revelation these results grant us one regarding the standard deviation of second order agents. In the default equal population the second order standard deviation is $\mathbf{18.307}$. Looking across the results we have so far in tables \hyperref[table:non-sig-zero-order-simple]{3.2}, \hyperref[table:non-sig-first-order-simple]{3.3} and \hyperref[table:non-sig-second-order-simple]{3.4}, it can be seen that the number of first order agents is inversely correlated with the standard deviation of second order agents and partially positively effected by the increased number of zero order agents. This is due to the fact that the second order agents have to model more first order agents, which leads to more variance in their behaviour due to the first order agents' stochastic behaviour. This is also reflected in the fact that the standard deviation of second order agents is the highest (above 40) in the 2 scenarios where the zero order to first order ratio is the lowest with a value of $1/2$ (in the zero order under-abundance and first order over-abundance scenarios). 