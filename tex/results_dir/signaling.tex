For the tables presented in this subsection, to denote the inital population, a similar format of:

\begin{equation}
    x_{s_{0}} / y_{s_{1}} / z_{s_{2}} / a_{r_{0}} / b_{r_{1}} / c_{r_{2}}
\end{equation}

\noindent is used, where $x$, $y$, and $z$ are the number of 0, 1, and 2 order signaling agents, and $a$, $b$, and $c$ are the number of 0, 1, and 2 order receiving agents respectively. The $s$ and $r$ subscripts denote signaling and receiving agents in an understandable format.

\subsubsection{Default Equal Population}

The default equal population consisted of 100 agents of each order, with each order being split into half signaling and half receiving agents. The end results are shown in \hyperref[table:sig-default-equal]{Table 3.5}, where the signaling agents are on the left and the receiving agents are on the right and are labelled as such.   

\begin{table}[h]
    \centering
    \begin{tabular}{|c|c|c|c|c|}
    \hline
    \multicolumn{5}{|c|}{$50_{s_{0}}/50_{s_{1}}/50_{s_{2}}/50_{r_{0}}/50_{r_{1}}/50_{r_{2}}$} \\
    \hline
    \multicolumn{1}{|c|}{} & \multicolumn{2}{|c|}{Signaling} & \multicolumn{2}{|c|}{Receiving} \\
    \hline
    Order & AOAS & Std Dev & AOAS & Std Dev \\
    \hline
    0     & 4.525   & 0.252    & 4.550   & 0.245   \\
    1     & 13.896  & 17.429   & 78.209  & 1.043   \\
    2     & 58.253  & 24.377   & 69.144  & 6.549   \\
    \hline
    \end{tabular}
    \caption{The average order agent scores for each order in the default equal population for the signaling simulation.}
    \label{table:sig-default-equal}
\end{table}

The overall order-wise results, which groups signaling and receiving agents together, are shown in \hyperref[table:sig-default-equal-overall]{Table 3.6}.

\begin{table}[h]
    \centering
    \begin{tabular}{|c|c|c|}
    \hline
    Order & AOAS & Std Dev \\
    \hline
    0     & 4.538   & 0.248    \\
    1     & 46.053  & 9.236    \\
    2     & 63.699  & 15.463   \\
    \hline
    \end{tabular}
    \caption{The average order agent scores for each order in the default equal population for the signaling simulation.}
    \label{table:sig-default-equal-overall}
\end{table}

The results in \hyperref[table:sig-default-equal]{Table 3.5} offer numerous insights and base conclusions that can be used in future ones. In terms of zero order agents, they have almost identical scores, which means their respective higher order agents either have no effect or coincidentally the same effect on their scores. In terms of first order agents, the signaling agents have a much lower score than the receiving agents, which have the leading score across all order/agent type combinations with $78.209$ AOAS. First order signaling agents also have high standard deviations, which means their behaviour is not very deterministic. In terms of second order agents, the receiving agents have a higher score than the signaling agents, but the difference is not as large as the first order agents. \hyperref[table:sig-default-equal-overall]{Table 3.6} tells us a different story, where compared to the scores of the non-signaling default equal population in \hyperref[table:non-sig-default-equal-simple]{Table 3.1}, there is a more equal distribution of the scores across orders and the overall average order agent score is lower. The score difference between the orders is $(\mathbf{sd^0_1 = 41.515}, \mathbf{sd^1_2 = 17.646})$



\subsubsection{Zero Order Over and Under Abundance}

The zero order over and under-abundance scenarios are the same as the zero order over and under-abundance scenarios in the non-signaling simulation, with each order being split into half signaling and half receiving agents. The end results are shown in \hyperref[table:sig-zero-order-simple]{Table 3.7}.

\begin{table}[h]
    \centering
    \begin{tabular}{|c|c|c|c|c|}
    \hline
    \multicolumn{5}{|c|}{$75_{s_{0}}/37_{s_{1}}/37_{s_{2}}/75_{r_{0}}/38_{r_{1}}/38_{r_{2}}$} \\
    \hline
    \multicolumn{1}{|c|}{} & \multicolumn{2}{|c|}{Signaling} & \multicolumn{2}{|c|}{Receiving} \\
    \hline
    Order & AOAS & Std Dev & AOAS & Std Dev \\
    \hline
    0     & 4.026   & 0.293    & 4.039   & 0.279   \\
    1     & 15.140  & 21.570   & 86.602  & 1.175   \\
    2     & 55.402  & 34.814   & 88.039  & 3.657   \\
    \hline
    \end{tabular}
    \qquad
    \begin{tabular}{|c|c|c|c|c|}
        \hline
        \multicolumn{5}{|c|}{$30_{s_{0}}/60_{s_{1}}/60_{s_{2}}/30_{r_{0}}/60_{r_{1}}/60_{r_{2}}$} \\
        \hline
        \multicolumn{1}{|c|}{} & \multicolumn{2}{|c|}{Signaling} & \multicolumn{2}{|c|}{Receiving} \\
        \hline
        Order & AOAS & Std Dev & AOAS & Std Dev \\
        \hline
        0     & 5.143   & 0.252    & 5.127   & 0.227   \\
        1     & 11.889  & 13.713   & 71.605  & 0.939   \\
        2     & 58.529  & 17.660   & 59.255  & 6.802   \\
        \hline
        \end{tabular}
    \caption{The average order agent scores for each order in the zero order over and under-abundance scenarios for the signaling simulation}
    \label{table:sig-zero-order-simple}
\end{table}

The overall order-wise results, which groups signaling and receiving agents together, are shown in \hyperref[table:sig-zero-order-overall]{Table 3.8}.

\begin{table}[h]
    \centering
    \begin{tabular}{|c|c|c|}
    \hline
    \multicolumn{3}{|c|}{$75_{s_{0}}/37_{s_{1}}/37_{s_{2}}/75_{r_{0}}/38_{r_{1}}/38_{r_{2}}$} \\
    \hline
    \multicolumn{1}{|c|}{} & \multicolumn{2}{|c|}{Overall} \\
    \hline
    Order & AOAS & Std Dev \\
    \hline
    0     & 4.033   & 0.286    \\
    1     & 50.871  & 11.372   \\
    2     & 71.721  & 18.921   \\
    \hline
\end{tabular}
\qquad
\begin{tabular}{|c|c|c|}
    \hline
    \multicolumn{3}{|c|}{$30_{s_{0}}/60_{s_{1}}/60_{s_{2}}/30_{r_{0}}/60_{r_{1}}/60_{r_{2}}$} \\
    \hline
    \multicolumn{1}{|c|}{} & \multicolumn{2}{|c|}{Overall} \\
    \hline
    Order & AOAS & Std Dev \\
    \hline
    0     & 5.135   & 0.240    \\
    1     & 41.747  & 7.326   \\
    2     & 58.892  & 12.231   \\
    \hline
\end{tabular}
\caption{The average order agent scores for each order in the zero order over and under-abundance scenarios for the signaling simulation}
\label{table:sig-zero-order-overall}
\end{table}

The results show that when there are more zero order agents, the total average order agent score increases with only the zero order agents experiencing a loss in AOAS. This is because the more zero order agents there are, the more ways of earning score for first and second order agents. The score difference between orders is $(\mathbf{sd^0_1 = 46.838}, \mathbf{sd^1_2 = 20.85})$ which shows the overall score increase compared to the default equal population. A notable performance increase can be seen in second order receiving agents, which surpass the first order signaling agents in AOAS. 

When there are less zero order agents, the total average order agent score decreases with only the zero order agents experiencing a gain in AOAS. This is in line with the over-abundance results as with less zero order agents the opposite order-wise effects are seen. The score difference between orders is $(\mathbf{sd^0_1 = 36.612}, \mathbf{sd^1_2 = 17.145})$ which shows the overall score decrease compared to the default equal population. The second order receiving agents perform better than how they do in the default equal population, which could be due to the fact that there are more first order agents, both signaling and receiving, for them to benefit from.

\subsubsection{First Order Over and Under Abundance}

The first order over and under-abundance scenarios are the same as the first order over and under-abundance scenarios in the non-signaling simulation, with each order being split into half signaling and half receiving agents. The end results are shown in \hyperref[table:sig-first-order-simple]{Table 3.9}.

\begin{table}[h]
    \centering
    \begin{tabular}{|c|c|c|c|c|}
    \hline
    \multicolumn{5}{|c|}{$37_{s_{0}}/75_{s_{1}}/37_{s_{2}}/38_{r_{0}}/75_{r_{1}}/38_{r_{2}}$} \\
    \hline
    \multicolumn{1}{|c|}{} & \multicolumn{2}{|c|}{Signaling} & \multicolumn{2}{|c|}{Receiving} \\
    \hline
    Order & AOAS & Std Dev & AOAS & Std Dev \\
    \hline
    0     & 3.991   & 0.269    & 3.972   & 0.263   \\
    1     & 13.860  & 18.372   & 86.269  & 1.062   \\
    2     & 57.635  & 32.302   & 70.885  & 10.415   \\
    \hline
    \end{tabular}
    \qquad
    \begin{tabular}{|c|c|c|c|c|}
        \hline
        \multicolumn{5}{|c|}{$60_{s_{0}}/30_{s_{1}}/60_{s_{2}}/60_{r_{0}}/30_{r_{1}}/60_{r_{2}}$} \\
        \hline
        \multicolumn{1}{|c|}{} & \multicolumn{2}{|c|}{Signaling} & \multicolumn{2}{|c|}{Receiving} \\
        \hline
        Order & AOAS & Std Dev & AOAS & Std Dev \\
        \hline
        0     & 4.526   & 0.269    & 4.532   & 0.244   \\
        1     & 13.275  & 17.989   & 71.823  & 0.901   \\
        2     & 49.782  & 24.620   & 78.312  & 2.252   \\
        \hline
    \end{tabular}
    \caption{The average order agent scores for each order in the first order over and under-abundance scenarios for the signaling simulation}
    \label{table:sig-first-order-simple}
\end{table}

The overall order-wise results, which groups signaling and receiving agents together, are shown in \hyperref[table:sig-first-order-overall]{Table 3.10}.

\begin{table}[h]
    \centering
    \begin{tabular}{|c|c|c|}
        \hline
        \multicolumn{3}{|c|}{$37_{s_{0}}/75_{s_{1}}/37_{s_{2}}/38_{r_{0}}/75_{r_{1}}/38_{r_{2}}$} \\
        \hline
        \multicolumn{1}{|c|}{} & \multicolumn{2}{|c|}{Overall} \\
        \hline
        Order & AOAS & Std Dev \\
        \hline
        0     & 3.982   & 0.266    \\
        1     & 50.064  & 9.717   \\
        2     & 64.260  & 13.426   \\
        \hline
    \end{tabular}
    \qquad
    \begin{tabular}{|c|c|c|}
        \hline
        \multicolumn{3}{|c|}{$60_{s_{0}}/30_{s_{1}}/60_{s_{2}}/60_{r_{0}}/30_{r_{1}}/60_{r_{2}}$} \\
        \hline
        \multicolumn{1}{|c|}{} & \multicolumn{2}{|c|}{Overall} \\
        \hline
        Order & AOAS & Std Dev \\
        \hline
        0     & 4.529   & 0.256    \\
        1     & 42.549  & 9.445   \\
        2     & 64.047  & 13.436   \\
        \hline
    \end{tabular}
    \caption{The average order agent scores for each order in the first order over and under-abundance scenarios for the signaling simulation}
    \label{table:sig-first-order-overall}
\end{table}

The results show that when there are more first order agents, the zero order score decreases and the first order score increases, with the second order score remaining relatively constant. The score difference between orders is $(\mathbf{sd^0_1 = 46.082}, \mathbf{sd^1_2 = 14.196})$. In this configuration, the only group with more score compared to the default equal population is first order receiving agents, which prove time and time again that they are the most advantageous ones in the signaling simulation in most cases. 

When there are less first order agents, the zero and first order scores decrease a bit, with the overall scores remaining relatively constant. The score difference between orders is $(\mathbf{sd^0_1 = 38.020}, \mathbf{sd^1_2 = 21.498})$. The behaviour of second order agents is particularly peculiar, as the signaling second order agents have less score but the receiving second order agents have more score. This could suggest that while the second order signaling agents can mainly benefit from first order agents, the second order receiving agents can benefit from both first and second order agents, signaling or receiving.

\subsubsection{Second Order Over and Under Abundance}

The second order over and under-abundance scenarios are the same as the second order over and under-abundance scenarios in the non-signaling simulation, with each order being split into half signaling and half receiving agents. The end results are shown in \hyperref[table:sig-second-order-simple]{Table 3.11}.

\begin{table}[h]
    \centering
    \begin{tabular}{|c|c|c|c|c|}
    \hline
    \multicolumn{5}{|c|}{$37_{s_{0}}/37_{s_{1}}/75_{s_{2}}/38_{r_{0}}/38_{r_{1}}/75_{r_{2}}$} \\
    \hline
    \multicolumn{1}{|c|}{} & \multicolumn{2}{|c|}{Signaling} & \multicolumn{2}{|c|}{Receiving} \\
    \hline
    Order & AOAS & Std Dev & AOAS & Std Dev \\
    \hline
    0     & 5.269    & 0.212     & 5.301     & 0.210   \\
    1     & 14.202   & 16.219    & 60.157    & 0.774   \\
    2     & 53.502   & 11.432    & 64.183    & 1.449   \\
    \hline
    \end{tabular}
    \qquad
    \begin{tabular}{|c|c|c|c|c|}
    \hline
    \multicolumn{5}{|c|}{$60_{s_{0}}/60_{s_{1}}/30_{s_{2}}/60_{r_{0}}/60_{r_{1}}/30_{r_{2}}$} \\
    \hline
    \multicolumn{1}{|c|}{} & \multicolumn{2}{|c|}{Signaling} & \multicolumn{2}{|c|}{Receiving} \\
    \hline
    Order & AOAS & Std Dev & AOAS & Std Dev \\
    \hline
    0     & 4.015   & 0.301    & 4.003   & 0.293   \\
    1     & 12.944  & 19.765   & 92.317  & 1.313   \\
    2     & 59.171  & 37.668   & 83.654  & 9.141   \\
    \hline
    \end{tabular}
    \caption{The average order agent scores for each order in the second order over and under-abundance scenarios for the signaling simulation}
    \label{table:sig-second-order-simple}
\end{table}
    
The overall order-wise results, which groups signaling and receiving agents together, are shown in \hyperref[table:sig-second-order-overall]{Table 3.12}.

\begin{table}[h]
    \centering
    \begin{tabular}{|c|c|c|}
    \hline
    \multicolumn{3}{|c|}{$37_{s_{0}}/37_{s_{1}}/75_{s_{2}}/38_{r_{0}}/38_{r_{1}}/75_{r_{2}}$} \\
    \hline
    \multicolumn{1}{|c|}{} & \multicolumn{2}{|c|}{Overall} \\
    \hline
    Order & AOAS & Std Dev \\
    \hline
    0     & 5.285   & 0.211    \\
    1     & 37.179  & 8.497   \\
    2     & 58.843  & 6.441   \\
    \hline
    \end{tabular}
    \qquad
    \begin{tabular}{|c|c|c|}
    \hline
    \multicolumn{3}{|c|}{$60_{s_{0}}/60_{s_{1}}/30_{s_{2}}/60_{r_{0}}/60_{r_{1}}/30_{r_{2}}$} \\
    \hline
    \multicolumn{1}{|c|}{} & \multicolumn{2}{|c|}{Overall} \\
    \hline
    Order & AOAS & Std Dev \\
    \hline
    0     & 4.009   & 0.297    \\
    1     & 52.630  & 10.539   \\
    2     & 71.413  & 23.404   \\
    \hline
    \end{tabular}
    \caption{The average order agent scores for each order in the second order over and under-abundance scenarios for the signaling simulation}
    \label{table:sig-second-order-overall}
\end{table}

The results show that when there are more second order agents, the zero order AOAS increases and the first & second order AOAS decreases, with the first order signaling agents having a small increase in score. The score difference between orders is $(\mathbf{sd^0_1 = 31.194}, \mathbf{sd^1_2 = 21.664})$. One notable phenomenon observed in this scenario is the noticably lower standard deviation in second order overall AOAS, which is lower than the value of 10 only in this scenario.

When there are less second order agents, the overall total AOAS jumps from $\mathbf{114.29}$ in the default equal population to $\mathbf{128.052}$. Other than the zero order agents, the other orders all have more overall AOAS, with the first order receiving agents once again being on the top with $\mathbf{92.317}$ AOAS. The score difference between orders is $(\mathbf{sd^0_1 = 48.621}, \mathbf{sd^1_2 = 18.783})$. The standard deviation of the second order overall AOAS is also the highest so far in this scenario, at $\mathbf{23.404}$.

\subsubsection{Signaling and Receiving Over-Abundance}

In the signaling and receving over-abundance scenarios, the orders have equal distribution of agents, like the default equal population scenario. The difference is that there is a $70/30$ ratio between signaling and receiving agents, and the inverse. The population percentages are $70_{s_{0}}/70_{s_{1}}/70_{s_{2}}/30_{r_{0}}/30_{r_{1}}/30_{r_{2}}$ and $30_{s_{0}}/30_{s_{1}}/30_{s_{2}}/70_{r_{0}}/70_{r_{1}}/70_{r_{2}}$. The end results are shown in \hyperref[table:sig-over-abundance]{Table 3.13}.


\begin{table}[h]
    \centering
    \begin{tabular}{|c|c|c|c|c|}
    \hline
    \multicolumn{5}{|c|}{$70_{s_{0}}/70_{s_{1}}/70_{s_{2}}/30_{r_{0}}/30_{r_{1}}/30_{r_{2}}$} \\
    \hline
    \multicolumn{1}{|c|}{} & \multicolumn{2}{|c|}{Signaling} & \multicolumn{2}{|c|}{Receiving} \\
    \hline
    Order & AOAS & Std Dev & AOAS & Std Dev \\
    \hline
    0     & 5.125     & 0.275    & 5.123      & 0.284   \\
    1     & 11.854    & 16.420   & 107.980    & 1.433   \\
    2     & 66.860    & 45.308   & 73.014     & 27.055   \\
    \hline
    \end{tabular}
    \qquad
    \begin{tabular}{|c|c|c|c|c|}
    \hline
    \multicolumn{5}{|c|}{$30_{s_{0}}/30_{s_{1}}/30_{s_{2}}/70_{r_{0}}/70_{r_{1}}/70_{r_{2}}$} \\
    \hline
    \multicolumn{1}{|c|}{} & \multicolumn{2}{|c|}{Signaling} & \multicolumn{2}{|c|}{Receiving} \\
    \hline
    Order & AOAS & Std Dev & AOAS & Std Dev \\
    \hline
    0     & 3.761   & 0.263    & 3.766   & 0.269   \\
    1     & 13.943  & 18.462   & 48.878  & 0.636   \\
    2     & 42.050  & 9.681    & 79.185  & 17.140   \\
    \hline
    \end{tabular}
    \caption{The average order agent scores for each order in the singaling and receiving over-abundance scenarios for the signaling simulation}
    \label{table:sig-over-abundance}
\end{table}

The overall order-wise results, which groups signaling and receiving agents together, are shown in \hyperref[table:sig-overall]{Table 3.14}.

\begin{table}[h]
    \centering
    \begin{tabular}{|c|c|c|}
    \hline
    \multicolumn{3}{|c|}{$70_{s_{0}}/70_{s_{1}}/70_{s_{2}}/30_{r_{0}}/30_{r_{1}}/30_{r_{2}}$} \\
    \hline
    \multicolumn{1}{|c|}{} & \multicolumn{2}{|c|}{Overall} \\
    \hline
    Order & AOAS & Std Dev \\
    \hline
    0     & 5.124   & 0.279    \\ 
    1     & 59.917  & 8.926   \\
    2     & 69.937  & 36.182   \\ 
    \hline
    \end{tabular}
    \qquad
    \begin{tabular}{|c|c|c|}
    \hline
    \multicolumn{3}{|c|}{$30_{s_{0}}/30_{s_{1}}/30_{s_{2}}/70_{r_{0}}/70_{r_{1}}/70_{r_{2}}$} \\
    \hline
    \multicolumn{1}{|c|}{} & \multicolumn{2}{|c|}{Overall} \\
    \hline
    Order & AOAS & Std Dev \\
    \hline
    0     & 3.764   & 0.266    \\
    1     & 31.410  & 9.549   \\ 
    2     & 60.618  & 13.411   \\ 
    \hline
    \end{tabular}
    \caption{The average order agent scores for each order in the signaling and receiving over-abundance scenarios for the signaling simulation}
    \label{table:sig-overall}
\end{table}

The results show that when there are more signaling agents, every group of agents have an increased AOAS barring the exception of the first order signaling agents, which have a lower score than the default equal population. The score difference between orders is $(\mathbf{sd^0_1 = 54.793}, \mathbf{sd^1_2 = 10.020})$. Notable phenomena are the first order receiving agents' AOS being highest in this and all scenarios with a score of $\mathbf{107.980}$, and the second order agents having a relatively high standard deviation in their AOAS, at $\mathbf{45.308}$ for signaling agents and $\mathbf{27.055}$ for the receiving agents respectively. 

When there are more receiving agents, all agent groups excluding the second order receiving agents and first order receiving agents see a drastic decrease in their scores, with the second order agents seeing a relative increase. The score difference between orders is $(\mathbf{sd^0_1 = 27.646}, \mathbf{sd^1_2 = 29.208})$. The second order receiving agents performing relatively good in both scenarios is notable, and might suggest that they'll perform relatively well in any given scenario.

